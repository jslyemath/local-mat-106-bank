\setvseed{55637}
\skillheader{W1}

\begin{enumerate}
    \item Convert $3,576,276$ to ancient Egyptian.

    \vspace{20pt}
    \begin{ansenv}
        $\Large\textpmhg{\Hmillion\Hmillion\Hmillion\HCthousand\HCthousand\HCthousand\HCthousand\HCthousand\HXthousand\HXthousand\HXthousand\HXthousand\HXthousand\HXthousand\HXthousand\Hthousand\Hthousand\Hthousand\Hthousand\Hthousand\Hthousand\Hhundred\Hhundred\Hten\Hten\Hten\Hten\Hten\Hten\Hten\Hone\Hone\Hone\Hone\Hone\Hone}$
    \end{ansenv}
    \vfill

    \item Convert the following ancient Babylonian numeral to modern Hindu-Arabic base ten.
    \begin{center}
        $\babt\babt\babt\babo\babo\babo\babo\babo\babo\hspace{30pt}\babt\babt\babo\babo\babo\hspace{30pt}\babt\babt\babt\babo\babo\babo\babo$
    \end{center}

    \vspace{20pt}
    \begin{ansenv}
        $131,014$
    \end{ansenv}
    \vfill
\end{enumerate}

\newpage

\skillheader{W1-E}
Compare and contrast the Roman numeral ``$\text{VII}$'' and the modern base-ten numeral ``511''. How are they the same? How are they different? What is fundamentally different about these two numeration systems that is key to understanding this comparison? Fully explain your answers.