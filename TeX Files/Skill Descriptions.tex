\setskilldesc{G1}{I can identify hypothetical student work as correct or incorrect, analyze the reasoning for incorrect mistakes, and make appropriate suggestions for improvement.}
\setskilldesc{G2}{I can solve word problems using basic arithmetic operations.}
\setskilldesc{W1}{I can convert the ancient Roman/Babylonian/Egyptian numeration systems to modern Hindu-Arabic base ten, and vice versa.}
\setskilldesc{W2}{I can compute addition and subtraction of multi-digit base-b whole numbers using non-standard addition and subtraction algorithms.}
\setskilldesc{W3}{I can identify uses of the associative, commutative, identity, zero product, and distributive properties of addition and multiplication.}
\setskilldesc{W4}{I can generate elementary word problems most suitable for each of the problem structures of addition, subtraction, multiplication, and division.}
\setskilldesc{W5}{I can compute multiplication of multi-digit base-b whole numbers using non-standard multiplication algorithms.}
\setskilldesc{W6}{I can compute the quotient of two whole or decimal numbers using standard and non-standard algorithms.}
\setskilldesc{N1}{I can identify divisibility statements as true or false.}
\setskilldesc{N2}{I can determine if a number is prime or composite via the prime testing method.}
\setskilldesc{N3}{I can compute the greatest common divisor (GCD) of two numbers using appropriate methods.}
\setskilldesc{N4}{I can compute the least common multiple (LCM) of two numbers using appropriate methods.}
\setskilldesc{F1}{I can convert fractions to different forms (including simplest form), and determine if two fractions are equivalent.}
\setskilldesc{F2}{I can construct both number line and area models for fractions.}
\setskilldesc{F3}{I can, given two rational numbers, find any number of rational numbers between them and, in the case of decimals, place these numbers accurately on a number line.}
\setskilldesc{F4}{I can compute the sum, difference, product, and quotient of mixed numbers.}
\setskilldesc{F5}{I can solve word problems using proportional reasoning.}
\setskilldesc{D1}{I can identify place values in decimal numbers using both numerical notation and English words, and represent decimal numbers using base-ten blocks.}
\setskilldesc{D2}{I can convert numbers between fractions, decimals, and percents.}
\setskilldesc{D3}{I can order any rational numbers, including decimals, from least to greatest.}
\setskilldesc{D4}{I can solve word problems using percents.}
\setskilldesc{G2-E}{I can explain my reasoning for each step of work in solving a word problem with the basic arithmetic operations, in language that supports elementary students' understanding of how to solve such problems.}
\setskilldesc{W1-E}{I can compare and contrast the ancient and modern numeration systems, and explain how they represent values. }
\setskilldesc{W3-E}{I can explain why the properties of addition and multiplication work, in terms that elementary students would understand. }
\setskilldesc{W4-E}{I can explain why word problems are most suitable for particular problem structures.}
\setskilldesc{N1-E}{I can explain why divisibility statements are logically correct or incorrect, using general logic that elementary students would understand.}
\setskilldesc{N2-E}{I can explain why the method for testing primes is designed the way it is, and why it works.}
\setskilldesc{F2-E}{I can explain how to construct fraction models, and how students can understand how they represent the numerator and denominator of a fraction.}
\setskilldesc{F3-E}{I can explain the denseness property of the rational numbers, by using language appropriate for an elementary student, and by using concrete examples.}
\setskilldesc{F5-E}{I can explain my reasoning for each step of work in solving a proportional reasoning word problem, in language that supports elementary students' understanding of how to solve such problems.}
\setskilldesc{D1-E}{I can explain the relationship between place values in the base ten decimal system, and how the base ten blocks help demonstrate that relationship.}
