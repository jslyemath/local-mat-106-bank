\setvseed{\VAR{seed}}
\skillheader{F3}

\begin{enumerate}
    \item Find \VAR{p1_how_many} rational numbers between $\VAR{p1_a}$ and $\VAR{p1_b}$. \VAR{num_line_directions}

    \vfill

    \setmyx{0}{0}
    \begin{tikzpicture}[scale=2.3]
        \draw [line width=2, Stealth-Stealth ] (-3.5,0) -- (3.5,0);
    \end{tikzpicture}

    \vfill
    \begin{ansenv}
        Examples:\newline
        \vspace{12pt}
        \noindent
        \VAR{p1_numbers}
        \noindent
    \end{ansenv}
    \vfill

    \item Find \VAR{p2_how_many} rational numbers between $\VAR{p2_a}$ and $\VAR{p2_b}$.

    \vfill
    \vfill
    \begin{ansenv}
        Examples:\newline
        \vspace{12pt}
        \noindent
        $\VAR{p2_numbers}$
    \end{ansenv}
    \vfill
    \vfill
    \vfill
\end{enumerate}

\newpage

\begin{setskill}{F3-E}
    I can explain the denseness property of the rational numbers, by using language appropriate for an elementary student, and by using concrete examples.
\end{setskill}

In language appropriate for an elementary student, what does the denseness property of the rational numbers say? Explain how one or both of your examples from the previous page demonstrates this property. Then, suppose the student challenged you to find \VAR{expl_how_many} numbers between the given two rational numbers. What would you do? (You do not need to write out \VAR{expl_how_many} individual numbers.)